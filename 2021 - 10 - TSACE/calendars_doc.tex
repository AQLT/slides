\PassOptionsToPackage{unicode=true}{hyperref} % options for packages loaded elsewhere
\PassOptionsToPackage{hyphens}{url}
%
\documentclass[10pt,xcolor=table,color={dvipsnames,usenames},ignorenonframetext,usepdftitle=false,french]{beamer}
\setbeamertemplate{caption}[numbered]
\setbeamertemplate{caption label separator}{: }
\setbeamercolor{caption name}{fg=normal text.fg}
\beamertemplatenavigationsymbolsempty
\usepackage{caption}
\captionsetup{skip=0pt,belowskip=0pt}
%\setlength\abovecaptionskip{-15pt}
\usepackage{lmodern}
\usepackage{amssymb,amsmath,mathtools,multirow}
\usepackage{float,hhline}
\usepackage{tikz}
\usepackage[tikz]{bclogo}
\usepackage{mathtools}
\usepackage{ifxetex,ifluatex}
\usepackage{fixltx2e} % provides \textsubscript
\ifnum 0\ifxetex 1\fi\ifluatex 1\fi=0 % if pdftex
  \usepackage[T1]{fontenc}
  \usepackage[utf8]{inputenc}
  \usepackage{textcomp} % provides euro and other symbols
\else % if luatex or xelatex
  \usepackage{unicode-math}
  \defaultfontfeatures{Ligatures=TeX,Scale=MatchLowercase}
\fi
\usetheme[coding=utf8,language=english,
,titlepagelogo=img/SACElogo
]{TorinoTh}
% use upquote if available, for straight quotes in verbatim environments
\IfFileExists{upquote.sty}{\usepackage{upquote}}{}
% use microtype if available
\IfFileExists{microtype.sty}{%
\usepackage[]{microtype}
\UseMicrotypeSet[protrusion]{basicmath} % disable protrusion for tt fonts
}{}
\IfFileExists{parskip.sty}{%
\usepackage{parskip}
}{% else
\setlength{\parindent}{0pt}
\setlength{\parskip}{6pt plus 2pt minus 1pt}
}
\usepackage{hyperref}
\hypersetup{
            pdftitle={rjd3modelling and documentation},
            pdfauthor={Alain Quartier-la-Tente},
            pdfborder={0 0 0},
            breaklinks=true}
\urlstyle{same}  % don't use monospace font for urls
\newif\ifbibliography
\usepackage{color}
\usepackage{fancyvrb}
\newcommand{\VerbBar}{|}
\newcommand{\VERB}{\Verb[commandchars=\\\{\}]}
\DefineVerbatimEnvironment{Highlighting}{Verbatim}{commandchars=\\\{\}}
% Add ',fontsize=\small' for more characters per line
\usepackage{framed}
\definecolor{shadecolor}{RGB}{248,248,248}
\newenvironment{Shaded}{\begin{snugshade}}{\end{snugshade}}
\newcommand{\AlertTok}[1]{\textcolor[rgb]{0.94,0.16,0.16}{#1}}
\newcommand{\AnnotationTok}[1]{\textcolor[rgb]{0.56,0.35,0.01}{\textbf{\textit{#1}}}}
\newcommand{\AttributeTok}[1]{\textcolor[rgb]{0.77,0.63,0.00}{#1}}
\newcommand{\BaseNTok}[1]{\textcolor[rgb]{0.00,0.00,0.81}{#1}}
\newcommand{\BuiltInTok}[1]{#1}
\newcommand{\CharTok}[1]{\textcolor[rgb]{0.31,0.60,0.02}{#1}}
\newcommand{\CommentTok}[1]{\textcolor[rgb]{0.56,0.35,0.01}{\textit{#1}}}
\newcommand{\CommentVarTok}[1]{\textcolor[rgb]{0.56,0.35,0.01}{\textbf{\textit{#1}}}}
\newcommand{\ConstantTok}[1]{\textcolor[rgb]{0.00,0.00,0.00}{#1}}
\newcommand{\ControlFlowTok}[1]{\textcolor[rgb]{0.13,0.29,0.53}{\textbf{#1}}}
\newcommand{\DataTypeTok}[1]{\textcolor[rgb]{0.13,0.29,0.53}{#1}}
\newcommand{\DecValTok}[1]{\textcolor[rgb]{0.00,0.00,0.81}{#1}}
\newcommand{\DocumentationTok}[1]{\textcolor[rgb]{0.56,0.35,0.01}{\textbf{\textit{#1}}}}
\newcommand{\ErrorTok}[1]{\textcolor[rgb]{0.64,0.00,0.00}{\textbf{#1}}}
\newcommand{\ExtensionTok}[1]{#1}
\newcommand{\FloatTok}[1]{\textcolor[rgb]{0.00,0.00,0.81}{#1}}
\newcommand{\FunctionTok}[1]{\textcolor[rgb]{0.00,0.00,0.00}{#1}}
\newcommand{\ImportTok}[1]{#1}
\newcommand{\InformationTok}[1]{\textcolor[rgb]{0.56,0.35,0.01}{\textbf{\textit{#1}}}}
\newcommand{\KeywordTok}[1]{\textcolor[rgb]{0.13,0.29,0.53}{\textbf{#1}}}
\newcommand{\NormalTok}[1]{#1}
\newcommand{\OperatorTok}[1]{\textcolor[rgb]{0.81,0.36,0.00}{\textbf{#1}}}
\newcommand{\OtherTok}[1]{\textcolor[rgb]{0.56,0.35,0.01}{#1}}
\newcommand{\PreprocessorTok}[1]{\textcolor[rgb]{0.56,0.35,0.01}{\textit{#1}}}
\newcommand{\RegionMarkerTok}[1]{#1}
\newcommand{\SpecialCharTok}[1]{\textcolor[rgb]{0.00,0.00,0.00}{#1}}
\newcommand{\SpecialStringTok}[1]{\textcolor[rgb]{0.31,0.60,0.02}{#1}}
\newcommand{\StringTok}[1]{\textcolor[rgb]{0.31,0.60,0.02}{#1}}
\newcommand{\VariableTok}[1]{\textcolor[rgb]{0.00,0.00,0.00}{#1}}
\newcommand{\VerbatimStringTok}[1]{\textcolor[rgb]{0.31,0.60,0.02}{#1}}
\newcommand{\WarningTok}[1]{\textcolor[rgb]{0.56,0.35,0.01}{\textbf{\textit{#1}}}}
% Prevent slide breaks in the middle of a paragraph:
\widowpenalties 1 10000
\raggedbottom
\AtBeginPart{
  \let\insertpartnumber\relax
  \let\partname\relax
  \frame{\partpage}
}
\AtBeginSection{
  \ifbibliography
  \else
    \begin{frame}{Sommaire}
    \tableofcontents[currentsection, hideothersubsections]
    \end{frame}
  \fi
}
\setlength{\emergencystretch}{3em}  % prevent overfull lines
\providecommand{\tightlist}{%
  %\setlength{\itemsep}{0pt}
  \setlength{\parskip}{0pt}
  }
\setcounter{secnumdepth}{0}

% set default figure placement to htbp
\makeatletter
\def\fps@figure{htbp}
\makeatother

\usepackage{wrapfig}
\usepackage{booktabs}
\usepackage{longtable}
\usepackage{array}
\usepackage{multirow}
\usepackage[table]{xcolor}
\usepackage{wrapfig}
\usepackage{float}
\usepackage{colortbl}
\usepackage{pdflscape}
\usepackage{tabu}
\usepackage{threeparttable}
\usepackage{threeparttablex}
\usepackage[normalem]{ulem}
\usepackage{makecell}
\usepackage{animate}
\usepackage{fontawesome5}

\title{rjd3modelling and documentation}
\ateneo{TSACE, October 26, 2021}
\author{Alain Quartier-la-Tente}
\date{}


\setrellabel{}

\setcandidatelabel{}

\rel{}
\division{Insee)}

\departement{\href{mailto:alain.quartier-la-tente@insee.fr}{\nolinkurl{alain.quartier-la-tente@insee.fr}}}
\makeatletter
\let\@@magyar@captionfix\relax
\makeatother

\DeclareMathOperator{\Cov}{Cov}
\newcommand{\E}[1]{\mathbb{E}\left[ #1 \right]}
\newcommand{\V}[1]{\mathbb{V}\left[ #1 \right]}
\newcommand{\cov}[2]{\Cov\left( #1\,,\,#2 \right)}

\begin{document}
\begin{frame}[plain,noframenumbering]
\titlepage
\end{frame}

\hypertarget{rjd3modelling}{%
\section{rjd3modelling}\label{rjd3modelling}}

\begin{frame}[fragile]{What is available in \texttt{rjd3modelling}?}
\protect\hypertarget{what-is-available-in-rjd3modelling}{}
Package available on GitHub:

\begin{Shaded}
\begin{Highlighting}[]
\NormalTok{remotes}\SpecialCharTok{::}\FunctionTok{install\_github}\NormalTok{(}\StringTok{"palatej/rjd3toolkit"}\NormalTok{)}
\NormalTok{remotes}\SpecialCharTok{::}\FunctionTok{install\_github}\NormalTok{(}\StringTok{"palatej/rjd3modelling"}\NormalTok{)}
\end{Highlighting}
\end{Shaded}

\pause

\begin{itemize}
\tightlist
\item
  create trading-days variables with a \textbf{user-defined calendar}:
  easter related days (\texttt{calendar.easter}), fixed days
  (\texttt{calendar.fixedday}) and from specific holidays
  \texttt{calendar.holiday} \faArrowCircleRight{} see
  \texttt{?calendar.new} for a complete example.
\end{itemize}

\pause

\begin{itemize}
\tightlist
\item
  create \textbf{common regressors}: stock trading days
  (\texttt{stock.td}), leap year regressors (\texttt{lp.variable}),
  easter regressors (\texttt{easter.variable}), outliers
  (\texttt{ao.variable}, \texttt{ls.variable}, \texttt{tc.variable},
  \texttt{so.variable}), ramp (\texttt{ramp.variable}), intervention
  variables (\texttt{intervention.variable}), periodic dummies
  (\texttt{periodic.dummies}) and contrast
  (\texttt{periodic.contrasts}), trigonometric variables
  (\texttt{trigonometric.variables})
\end{itemize}

\pause

\bcquestion How do you get started with these tools?
\end{frame}

\hypertarget{documentation}{%
\section{Documentation}\label{documentation}}

\hypertarget{static-document}{%
\subsection{Static document}\label{static-document}}

\begin{frame}{Vignette/Word/PDF documentation}
\protect\hypertarget{vignettewordpdf-documentation}{}
Usual documentation, already available for JDemetra+

\animategraphics[loop, autoplay, width=\linewidth]{2}{img/gif/docx/docx_}{1}{8}
\end{frame}

\begin{frame}[fragile,allowframebreaks]{HTML/PDF tutorials with
\texttt{unilur}}
\protect\hypertarget{htmlpdf-tutorials-with-unilur}{}
Use \texttt{unilur} (github.com/koncina/unilur) to create
tutorials/practicals or examination papers with \texttt{rmarkdown}

\begin{verbatim}
---
output:
  unilur::tutorial_html: default # without solutions
  unilur::tutorial_html_solution: default # with solutions
  unilur::tutorial_pdf: default
  unilur::tutorial_pdf_solutionn: default
---
\end{verbatim}

\begin{Shaded}
\begin{Highlighting}[]
\InformationTok{\textasciigrave{}\textasciigrave{}\textasciigrave{}\{r, solution = TRUE\}}
\InformationTok{frenchCalendar \textless{}{-} calendar.new()}
\InformationTok{\textasciigrave{}\textasciigrave{}\textasciigrave{}}
\end{Highlighting}
\end{Shaded}

\animategraphics[loop, autoplay, width=\linewidth]{2}{img/gif/unilur/unilur_}{1}{23}
\end{frame}

\hypertarget{dynamic-tutorial}{%
\subsection{Dynamic tutorial}\label{dynamic-tutorial}}

\begin{frame}[fragile,allowframebreaks]{Create interactive tutorials
with \texttt{learnr}}
\protect\hypertarget{create-interactive-tutorials-with-learnr}{}
Tutorials includes in a R package
(\texttt{remotes::install\_github("AQLT/rjd3tutorials")})

\begin{Shaded}
\begin{Highlighting}[]
\InformationTok{\textasciigrave{}\textasciigrave{}\textasciigrave{}\{r regressors, exercise = TRUE\}}
\InformationTok{frenchCalendar \textless{}{-} calendar.new()}
\InformationTok{\textasciigrave{}\textasciigrave{}\textasciigrave{}}

\InformationTok{\textasciigrave{}\textasciigrave{}\textasciigrave{}\{r regressors{-}hint\}}
\InformationTok{\# define Saturday and Sunday as contrast}
\InformationTok{groups \textless{}{-} c(1, 1, 1, 1, 1, 0, 0)}
\InformationTok{\textasciigrave{}\textasciigrave{}\textasciigrave{}}

\InformationTok{\textasciigrave{}\textasciigrave{}\textasciigrave{}\{r regressors{-}solution\}}

\InformationTok{\textasciigrave{}\textasciigrave{}\textasciigrave{}}
\end{Highlighting}
\end{Shaded}

\animategraphics[loop, autoplay, width=\linewidth]{2}{img/gif/learnr/learnr_}{1}{44}
\end{frame}

\begin{frame}{Thank you for your attention}
\protect\hypertarget{thank-you-for-your-attention}{}
\begin{columns}
\begin{column}{0.4\textwidth}
\faIcon{r-project} packages:
\begin{itemize}
\item \href{https://github.com/palatej/rjd3toolkit}{\faGithub{} palatej/rjd3toolkit}

\item \href{https://github.com/palatej/rjd3modelling}{\faGithub{} palatej/rjd3modelling}
\end{itemize}
\end{column}
\begin{column}{0.6\textwidth}
Tutorials:
\begin{itemize}
\item \href{https://github.com/AQLT/rjd3tutorials}{\faGithub{} AQLT/rjd3tutorials}

\item example with unilur: \href{https://aqlt-formation-rte.netlify.app/TP/Enonces/R-2-CJO_solution.html}{https://aqlt-formation-rte.netlify.app/TP/Enonces/R-2-CJO\_solution.html}
\end{itemize}
\end{column}
\end{columns}

\vfill

Contact:
\href{mailto:alain.quartier-la-tente@insee.fr}{\faEnvelope{} alain.quartier-la-tente [at] insee.fr}
\end{frame}

\end{document}
